
\documentclass[11pt,a4paper]{article}
\usepackage[utf8]{inputenc}
%\usepackage[icelandic]{babel}
%\usepackage[T1]{fontenc}
\usepackage{amsmath}
\usepackage{amsfonts}
\usepackage{amssymb}
%\usepackage{graphicx}
%\author{Arnar Ingi Halldórsson}
%\title{Linear Motion}


%\documentclass{article}
\usepackage{graphicx}
\graphicspath{ {myndir/} }
\usepackage[T1]{fontenc} 
\usepackage[english]{babel}

\usepackage[utf8]{inputenc} 
\usepackage{graphics}
%\usepackage[pdftex]{graphicx}

\usepackage{caption}
\usepackage{subcaption}

\usepackage{titling}

\setlength{\droptitle}{-10em}
   
\title{Assignment 1 \\ T-509-RAFT \\ Electronics} % Title

\author{Arnar Ingi Halldórsson \\ Hjörleifur G Bergsteinsson \\ Snorri Stefánsson} % Author name
\begin{document}

\maketitle % Insert the title, author and date

\begin{tabular}{lr}
Due date: 22.02.2015 \\
Teachers:\qquad Slawomir Koziel\\ % Instructor/supervisor
\qquad \qquad \qquad Adrian Bekasiewicz
\end{tabular}

\setlength\parindent{0pt} % Removes all indentation from paragraphs

\renewcommand{\labelenumi}{\alph{enumi}.} % Make numbering in the enumerate environment by letter rather than number (e.g. section 6)

\section*{Introduction}


\section*{Task 1: Perfomance Parameters of Op Amp}

\begin{enumerate}
  \item[1.]
  The circuit diagram shown below was simulated in $MultiSim$ to find the slew-rate(SR) of a op amp.\\
    \begin{minipage}{\linewidth}
      \centering       
       \includegraphics[width=10cm]{Task1-1Circuit}\\
   	 \captionof{figure}{yoyo} 
    \end{minipage}
    
Fig 2 shows the measurements from the Oscilloscope. The diagram shows the effect of slew rate limiting on the output rectangular waveform.\\
    \begin{minipage}{\linewidth}
      \centering       
       \includegraphics[width=10cm]{Task1-1-Oscilloscope}\\
      \captionof{figure}{} 
    \end{minipage}
		% \includegraphics[width=10cm]{Task1-1Circuit}\\
		% \includegraphics[width=10cm]{Task1-1-Oscilloscope}\\
  
  \item[2.]
	The circuit diagram shown below was simulated with AC analysis to determine the gain-bandwidth product(GBW) of the op amp.  
  
      \begin{minipage}{\linewidth}
      \centering
        \includegraphics[width=10cm]{Task1-2-Circuit}\\
    \captionof{figure}{} 
    \end{minipage}
    
    Fig 4 shows the measurments from the AC analysis of the circuit. 
    $$ BW \simeq f_{H} $$
    $$ GB = |A_{G}|BW $$
    
    We know that the gain is 1 of a voltage follower.
    $$ GB = BW $$
    
    The $f_{H}$ is 995.486 kHz in Fig 4. So the gain-bandwidth product is
    $$ GB = 995.486  kHz$$
    
    \begin{minipage}{\linewidth}
      \centering
        \includegraphics[width=10cm]{Task1-2-ACAnalysis}\\
    \captionof{figure}{}     
    \end{minipage}
  % 		\includegraphics[width=10cm]{Task1-2-Circuit}\\
		% \includegraphics[width=10cm]{Task1-2-ACAnalysis}\\
  \item[3.]
  
  
  The circuit below was simulated in MultiSim to determine the DC differential gain $A_{d}$ and the input offset voltage $V_{OS}$ of the op amp.
  
      \begin{minipage}{\linewidth}
      \centering
      \includegraphics[width=10cm]{Task1-3-Circuit}\\
    \captionof{figure}{}   
    \end{minipage}
    
Fig 6 show the result of the DC analysis of the circuit in Fig 5. Hence that the DC differential gain is
$$A_{d} = 206.505 \frac{V}{mV} $$    
    
    
        \begin{minipage}{\linewidth}
      \centering
      \includegraphics[width=10cm]{Task1-3-DCAnalysis}\\
    \captionof{figure}{}    
    \end{minipage}\\
    
Fig 7 shows the result of the DC analysis and hence the offset Voltage can be determine to be
$$ V_{OS} = 1.147 mV $$\\
    
    
\begin{minipage}{\linewidth}
	\centering
    \includegraphics[width=10cm]{Task1-3-Vos}\\
    \captionof{figure}{}    
    \end{minipage}\\
    
    
 \item[4.]
 

In table 1 there you can see the diffrence between theoretical values and simulated values of the $\mu A741$ op amp.\\

\begin{minipage}{\linewidth}

\begin{center}
	\begin{tabular}{l*{3}{c}r}{Hello}
		$\mu$ A741 op-Amp              & DataSheet & Simulation  \\
		\hline
		Slew Rate [$SR$] & 0.5 $ [\frac{V}{\mu s}]$ & 0.456 $ 		 [\frac{V}{\mu s}]$ \\
	Gain-Bandwidth [$GBW$]& 1 $ [MHz]$  & 995.487 $ [MHz]$ \\
	DC Differential Gain [$A_{d}$] & 200 $[\frac{V}{mV}]$ & 206.505 $[\frac{V}{mV}]$ \\
	Input Offset Voltage [$V_{os}$] & 5 $[mV]$& 1.147 $[mV]$ \\
		\end{tabular}
		\captionof{table}{Theoretical and simulated values of $\mu$A741 op amp}
\end{center}
\end{minipage}
\end{enumerate}

\section*{Task 2: Frequency Response of Inverting Amplifier}

\begin{enumerate}
  \item[1.]
  
  Figure below shows the inverting amplifier that we simulated in MultiSim. The task was to determine the 3dB corner frequency of the op amp with AC analysis.\\
  \begin{minipage}{\linewidth}
	\centering
  		\includegraphics[width=10cm]{Task2-1-Circuit}\\
    \captionof{figure}{}    
    \end{minipage}\\

  \item[2.]
  
  
    \begin{minipage}{\linewidth}
	\centering
  		\includegraphics[width=10cm]{Task2-1-ACAnalysis}\\
    \captionof{figure}{}    
    \end{minipage}\\
\end{enumerate}

\section*{Task 3: Op Amplifies Design}

\begin{enumerate}
  \item[1.]
  
  \item[2.]
  
  \item[3.]
  
\end{enumerate}

\section*{Task 4: Half - Wave Rectifier}


\begin{enumerate}
  \item[$\bold{1.}$]
  Here is the half-wave rectifier circuit:
  \\
  
	\begin{minipage}{\linewidth}
    	\centering
		\captionof{figure}{}        
        \includegraphics[width=13cm]{4_1.jpg}
    \end{minipage}
    
  
  \item[$\bold{2.}$]
  Now lets do a transient analysis of the circuit:
  \\
	\begin{minipage}{\linewidth}
    	\centering
		\captionof{figure}{Transient analysis of circuit the circuit in the last figure.}        
        \includegraphics[width=13cm]{4_2.jpg}
    \end{minipage}


    The data for the cursors on the previous figure:\\
    
    \begin{minipage}{\linewidth}
    	\centering
		\captionof{figure}{}        
        \includegraphics[width=9cm]{table_4_1.png}
    \end{minipage}
    
    \vspace{2em}
        
    So the peak voltage is y2 in the table above. That is $V_p = 9.935 V$ 
    
    The theoretical value was defined as $V_p = 10 V$ \\
    
    \begin{minipage}{\linewidth}
    	\centering
		\captionof{figure}{}        
        \includegraphics[width=13cm]{4_3.jpg}
    \end{minipage}
    
    \begin{minipage}{\linewidth}
    	\centering
		\captionof{figure}{}        
        \includegraphics[width=9cm]{table_4_2.png}
    \end{minipage}
    
    \vspace{2em}
    
	The ripple voltage can be calculated from the graph. The maximum difference in voltage over the diode over one period is the ripple voltage so it can simply be seen from the graph and table, using the cursors: $$ V_r = 9.5798 V - 7.4841 V = 2.096 V$$
    The theoretical ripple voltage value can be calculated with: $$ V_r = \dfrac{V_p}{fCR} = \dfrac{10}{50 \cdot 100 \cdot 10^{-6} \cdot 10^3} = 2 V$$
    \vspace{2em}
  
  \item[$\bold{3.}$]
  
  	The ripple voltage will then be: $V_r = V_p \cdot 0.02 = 0.1987 V$. Now we can calculate the maximum value of the capacitance, C: $$ C = \dfrac{V_p}{f R V_r} = \dfrac{V_p}{f R V_p \cdot 0.02} = \dfrac{1}{f R \cdot 0.02} = \dfrac{1}{50 \cdot 10^3 \cdot 0.02} = 1mF $$
  
  \begin{minipage}{\linewidth}
    	\centering
		\captionof{figure}{The transient analysis were the capacitor value is $1mF$ instead of $100\mu F$}        
%        \includegraphics[width=13cm]{4_4.jpg}
  \end{minipage}
    
   The ripple voltage should not be larger than 2 $\%$ of the peak voltage, $V_r(max) = 0.02 \cdot 9.935 V = 0.1987 V$. And from the graph and cursors the actual ripple voltage after the change: $$ V_r = 9.2364 \: V - 9.0928 \: V = 0.1436 \: V < V_r(max) = 0.1987 \: V$$
   

\end{enumerate}



\end{document}
\\
