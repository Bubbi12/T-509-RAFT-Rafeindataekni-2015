\subsection*{2.1}
  % Implement the CE-CC amplifier shown below:
    \begin{figure}[h!]
        \centering
        \includegraphics[width=13cm]{fig2-1.png}
        \captionof{figure}{The CE-CC amplifier to be implemented in Task 2.1}
    \end{figure}    

Figure X was implemented in $MultiSim$ and simulation was done.

\subsection*{2.2}
  % Using  proper  simulation  techniques,  determine  the  following  parameters  of  the  circuit:
  \subsection*{(i) Midband voltage gain}

	\begin{figure}[h]
        \centering
        \begin{subfigure}[h]{0.7\textwidth}
                \includegraphics[width=\textwidth]{Task2_1A.png}
                \label{fig:HJÖRLEIFUR}
        \end{subfigure}
        \begin{subfigure}[h]{0.25\textwidth}
                \includegraphics[width=\textwidth]{Task2_1B.png}
                \label{fig:LÁRUS}
        \end{subfigure}
        \captionof{figure}{AC analysis of circuit in figure X:Finding the midband gain}
	\end{figure}

    By using results from the AC analysis in $Multisim$ on figure X to determine the midband gain in decibel and convert over to gain($\frac{V}{V}$):

   	From figure X hence midband gain in decibel is:
   	$$ V_{0} = 32.1848 dB $$ 

   	$$ 20 \cdot log{\ A_{M}} = 32.1848 dB \rightarrow A_{M} = 40.66\  \frac{V}{V} $$



	\subsection*{(ii) Input resistance}
    \begin{figure}[h!]
        \centering
        \includegraphics[width=13cm]{Task2_2C.png}
        \captionof{figure}{}
    \end{figure}  

	\begin{figure}[h]
        \centering
        \begin{subfigure}[h]{0.7\textwidth}
                \includegraphics[width=\textwidth]{Task2_2A.png}
                \label{fig:}
        \end{subfigure}
        \begin{subfigure}[h]{0.25\textwidth}
                \includegraphics[width=\textwidth]{Task2_2B.png}
                \label{fig:}
        \end{subfigure}
        \captionof{figure}{}
	\end{figure}

	By adding a $R_{s}$ in the circuit as seen in figure X,  we can find the input resistor ($R_{s}$):
	$$ 20 \cdot log(A_{M2}) = 25.2002 dB  \rightarrow A_{M3} = 18.19 \frac{V}{V} $$
	By using result from Task2.2.i that the midband gain withot $R_{s}$ is $A_{M1} = 40.66 \frac{V}{V}$ then the formula for midband gain with $R_{s}$ would result in:

	$$ A_{M2} = A_{M1} \cdot \frac{R_{in}}{R_{in} + R_{s}} \rightarrow R_{in} = \frac{R_{s}}{\frac{A_{M1}}{A_{M2}}-1} $$
	
	$$R_{in} = \frac{6\ kHz}{\frac{40.66}{18.19}-1} = 4857.14 \Omega \ $$ \\

 	\subsection*{(iii) Output resistance}

	\begin{figure}[h!]
	    \centering
	    \includegraphics[width=13cm]{Task2_3C.png}
	    \captionof{figure}{}
	\end{figure}  \

	\begin{figure}[h]
	    \centering
	    \begin{subfigure}[h]{0.7\textwidth}
	            \includegraphics[width=\textwidth]{Task2_3A.png}
	            \label{fig:}
	    \end{subfigure}
	    \begin{subfigure}[h]{0.25\textwidth}
	            \includegraphics[width=\textwidth]{Task2_3B.png}
	            \label{fig:}
	    \end{subfigure}
	    \captionof{figure}{}
	\end{figure}

		By removing the load resistor and find the midband gain in the circuit as seen in figure X.

		$$ 20 \cdot log(A_{M3}) = 32.3419 dB  \rightarrow A_{M3} = 41.40 \frac{V}{V}$$


	By using result from Task2.2.i the midband gain with $R_{L}$ is $A_{M1} = 40.66$ and the formula for midband gain without $R_{L}$ is:

	$$A_{M1} = A_{M3} \cdot \frac{R_{L}}{R_{L} + R_{out}} \rightarrow R_{out} = R_{L} \cdot (\frac{A_{M3}}{A_{M1} - 1})  $$
	$$R_{out} = 500 \cdot (\frac{41.40}{40.66}-1) = 9.18\ \Omega$$ \\


	\subsection*{(iv) Lower 3dB frequency}

	\begin{figure}[h!]
	    \centering
	    \begin{subfigure}[h]{0.7\textwidth}
	            \includegraphics[width=\textwidth]{Task2_4A.png}
	            \label{fig:}
	    \end{subfigure}
	    \begin{subfigure}[h]{0.25\textwidth}
	            \includegraphics[width=\textwidth]{Task2_4B.png}
	            \label{fig:}
	    \end{subfigure}
	    \captionof{figure}{}
	\end{figure}

	To find the lower 3 dB frequency by finding the 3 db drop to the left from the midband gain as seen in figure X.

	$$f_{L} =  41.15 Hz$$




	
	\subsection*{(v) Upper 3dB frequency}

	\begin{figure}[h!]
	    \centering
	    \begin{subfigure}[h]{0.7\textwidth}
	            \includegraphics[width=\textwidth]{Task2_5A.png}
	            \label{fig:}
	    \end{subfigure}
	    \begin{subfigure}[h]{0.25\textwidth}
	            \includegraphics[width=\textwidth]{Task2_5B.png}
	            \label{fig:}
	    \end{subfigure}
	    \captionof{figure}{}
	\end{figure}
	To find the higher 3 dB frequency by finding the 3 db drop to the right from the midband gain as seen in figure X.

	$$f_{H} = 13.11 \ M Hz $$

  \subsection*{(vi) Output voltage when total harmonics distortion (THD) < 5\%}
  In order to evaluate the THD in the circuit, a fourier analysis has to be done. The input voltage is lowered gradually from 0.01 V until the THD is lower than 5\%. When that value was reached for the THD the Output voltage was determined 2.5471 V and the input voltage was then at 0.0731 V.
  \begin{figure}[h!]
        \centering
        \includegraphics[width=16cm]{Task2_6_MaxOutPutVoltage.png}
        \captionof{figure}{The fourier analysis with THD $\approx$ 5\%}
    \end{figure}
    
\pagebreak
  
\subsection*{2.3}
  % Summarize  the  circuit  parameters  in  a  table  and  attach  relevant  plots.  Explain  the
  % simulation techniques that you exploited.
  