
\documentclass[11pt,a4paper]{article}
\usepackage[utf8]{inputenc}

\usepackage{amsmath}
\usepackage{amsfonts}
\usepackage{amssymb}
\usepackage{graphicx}
\graphicspath{ {myndir/} }

\usepackage[T1]{fontenc} 
\usepackage[english]{babel}

\usepackage[utf8]{inputenc} 
\usepackage{graphics}
%\usepackage[pdftex]{graphicx}

\usepackage{caption}
\usepackage{subcaption}
\usepackage{subfiles}
\usepackage[top=2in, bottom=1.5in, left=1in, right=1in]{geometry}
   
\title{Laboratory Excercise 2 \\ T-509-RAFT \\ Electronics} % Title

\author{Arnar Ingi Halldórsson \\ Hjörleifur G. Bergsteinsson \\ Snorri Stefánsson} % Author name
\begin{document}
\maketitle % Insert the title, author and date

\begin{tabular}{lr}
    Due date: 08.03.2015 \\
    Teachers:\qquad Slawomir Koziel\\ % Instructor/supervisor
\qquad \qquad \qquad Adrian Bekasiewicz
\end{tabular}

\setlength\parindent{0pt} % Removes all indentation from paragraphs

\renewcommand{\labelenumi}{\alph{enumi}.} % Make numbering in the enumerate environment by letter rather than number (e.g. section 6)

\section*{Introduction}

    The  purpose  of  this  exercise  is to  examine  basic  diode  and  operational  amplifier  circuits. The exercise consists of several tasks; each of them requires performing the following steps:\\
        1. Setting up a circuit,\\
        2. Setting up supply voltages/currents and/or input signals, \\
        3. Preparing measuring devices, \\
        4. Taking measurements, \\
        5. Storing the results. \\

\pagebreak

\section*{Task 1: Half-Wave Rectifier}
    \subfile{task1}
\pagebreak
\section*{Task 2: Parameters of Operational Amplifier LM741}
    \subfile{task2}
\end{document}